\documentclass[a4paper,10pt]{report}
\usepackage[utf8x]{inputenc}
\usepackage[T1]{fontenc}
\usepackage[french]{babel} 
\usepackage{lmodern} % Pour changer le pack de police
\usepackage{makeidx}
\title{Projet 3A\\Analyse d'imagerie polarimétrique}
\author{\textsc{Guinaudeau} Alexandre\\
	\and 
	\textsc{Hulot} Pierre
	\and 
	\textsc{Dejoie} Etienne	
	}
\date{\today}
\makeindex
\begin{document}

\maketitle

\begin{abstract}
Le résumé (abstract en anglais) de mon article.
\end{abstract}

\chapter{contexte}
\section{ADM Polar, contexte du projet}
\section{Presentation des données}
 A mettre ici : présentation du set données. Nombre de pièces différentes. taille des images. Explication de ce qu'est la matrice de Muller


\chapter{Le traitement des données}

\section{Prétraitement des données}
Les différents types de prétraitements que l'on peut faire avant de traiter les données

\section{Les différentes approches de traitement des données}

\subsection{Réduction de dimension}
\subsubsection{PCA}
\paragraph{rappel de laméthode }
L'Analyse en Composantes Principales (ou PCA) consiste à essayer de représenter les données dans un espace de plus petites dimensions. Les vecteurs directeurs du nouvel espace maximise la variance entre les données. Nous présentons ici les résultats pour la dimension 2.
\paragraph{prétraitement utilisé}
Nous effectuaons cette PCA sur les centres des clusters préalablement présentés (cf 1.1.1)
\paragraph{résultats (notamment graphique)}
\paragraph{explication}
\paragraph{piste d'amélioration}

\subsection{Méthode de classification}

\subsubsection{Arbre décisionnel et Random Forest}
\paragraph{rappel de laméthode }
\paragraph{prétraitement utilisé}
\paragraph{résultats (notamment graphique)}
\paragraph{explication}
\paragraph{piste d'amélioration}

\subsubsection{K plus proche voisin}
\paragraph{rappel de laméthode }
\paragraph{prétraitement utilisé}
\paragraph{résultats (notamment graphique)}
\paragraph{explication}
\paragraph{piste d'amélioration}

\tableofcontents


Bla\index{bla} bla bla

\listoffigures
\listoftables
\printindex
\end{document}
